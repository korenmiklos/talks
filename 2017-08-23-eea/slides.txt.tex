\documentclass[compress,mathserif]{beamer}

%\usepackage[absolute]{textpos}
%\documentclass[handout,compress,mathserif]{beamer}
%\setbeameroption{show notes}

% This file is a solution template for:

% - Talk at a conference/colloquium.
% - Talk length is about 20min.
% - Style is ornate.



% Copyright 2004 by Till Tantau <tantau@users.sourceforge.net>.
%
% In principle, this file can be redistributed and/or modified under
% the terms of the GNU Public License, version 2.
%
% However, this file is supposed to be a template to be modified
% for your own needs. For this reason, if you use this file as a
% template and not specifically distribute it as part of a another
% package/program, I grant the extra permission to freely copy and
% modify this file as you see fit and even to delete this copyright
% notice.


\mode<presentation>
{
%  \usetheme{pittsburgh}
  % or ...

  \setbeamercovered{invisible}
  % or whatever (possibly just delete it)
}


\usepackage[USenglish]{babel}
\usepackage[utf8]{inputenc}




\renewcommand{\cite}[1]{({\small #1})}


\pretolerance5000 \hyphenpenalty9999
%\setlength{\TPHorizModule}{0.5cm} \setlength{\TPVertModule}{0.5cm}
%\textblockorigin{20mm}{20mm} % start everything near the top-left corner

\newcounter{ora}
\newcounter{perc}
\newcounter{kezdoora}
\newcounter{kezdoperc}
\newcounter{percek}
\setcounter{percek}{15}
\setcounter{kezdoora}{4} % for 1.35pm as the starting time

\providecommand{\leadingzero}[1]{\ifthenelse{\value{#1}<10}{0\arabic{#1}}{\arabic{#1}}}
\providecommand{\oradisplay}[1]{\ifthenelse{\value{#1}<60}{\arabic{kezdoora}:\leadingzero{#1}}{\setcounter{perc}{\value{#1}}\addtocounter{perc}{-60}\setcounter{ora}{\value{kezdoora}}\addtocounter{ora}{1}\arabic{ora}:\leadingzero{perc}}}

\providecommand{\notes}[1]{{\tiny\textbf{Note:} #1}}

%%\renewcommand{\alert}[1]{{\color{blue}#1}}
%%%%%%%%%%%%%%%%%%%%%%%%%%%%%%%%%%%%%%%%%%%%%%%%
%% Hasznos matek makrok
%%%%%%%%%%%%%%%%%%%%%%%%%%%%%%%%%%%%%%%%%%%%%%%%

\newcommand{\QED}{{}\hfill$\Box$}
\newcommand{\intl}[4]{\int_{#1}^{#2} \! {#3} \, \mathrm d{#4}}
\newcommand{\period}{\text{.}} % Ez azert kell, mert a matek . mashogy nez ki, mint a szovege.
\newcommand{\comma}{\text{,}}  % Ez azert kell, mert a matek , mashogy nez ki, mint a szovege.
\newcommand{\dist}{\,\mathop{\operatorname{\sim\,}}\limits}
\newcommand{\D}{\,\mathop{\operatorname{d}}\!}
%\newcommand{\E}{\mathop{\operatorname{E}}\nolimits}
\newcommand{\Lag}{\mathop{\operatorname{L}}}
\newcommand{\plim}{\mathop{\operatorname{plim}}\limits_{T\to\infty}\,}
\newcommand{\CES}[3]{\mathop{\operatorname{CES}}\left(\left\{#1\right\},\left\{#2\right\},#3\right)}
\newcommand{\cestwo}[5]{\left[#1^\frac1{#5}\,#2^\frac{#5-1}{#5}+#3^\frac1{#5}\,#4^\frac{#5-1}{#5}\right]^\frac{#5}{#5-1}}
\newcommand{\cesmore}[4]{\left[\sum_{#3}#1_{#3}^\frac1{#4}\,{#2}_{#3}^\frac{#4-1}{#4}\right]^\frac{#4}{#4-1}}
\newcommand{\cesPtwo}[5]{\left[#1\,#2^{1-#5}+#3\,#4^{1-#5}\right]^\frac{1}{1-#5}}
\newcommand{\cesPmore}[4]{\left[\sum_{#3}#1_{#3}\,#2_{#3}^{1-#4}\right]^\frac{1}{1-#4}}
\newcommand{\diff}[2]{\frac{\D #1}{\D #2}}
\newcommand{\pdiff}[2]{\frac{\partial #1}{\partial #2}}
\newcommand{\convex}[2]{\lambda #1 + (1-\lambda)#2}
\newcommand{\ABS}[1]{\left| #1 \right|}
\newcommand{\suchthat}{:\hskip1em}
\newcommand{\dispfrac}[2]{\frac{\displaystyle #1}{\displaystyle #2}} % Emeletes tortekhez hasznos.

\newcommand{\diag}{\mathop{\mathrm{diag\mathstrut}}}
\newcommand{\tr}{\mathop{\mathrm{tr\mathstrut}}}
\newcommand{\E}{\mathop{\mathrm{E\mathstrut}}}
\newcommand{\Var}{\mathop{\mathrm{Var\mathstrut}}\nolimits}
\newcommand{\Cov}{\mathop{\mathrm{Cov\mathstrut}}}
\newcommand{\sgn}{\mathop{\operatorname{sgn\mathstrut}}}

\newcommand{\covmat}{\mathbf\Sigma}
\newcommand{\ones}{\mathbf 1}
\newcommand{\zeros}{\mathbf 0}
\newcommand{\BAR}[1]{\overline{#1}}

\renewcommand{\time}[1]{\addtocounter{percek}{#1}}

\newlength{\tempsep}

\newenvironment{subeqs}{\setlength{\tempsep}{\arraycolsep}
\setlength{\arraycolsep}{0.13889em} % Ez azert kell, hogy ne hagyjon tul sok helyet az = korul.
\begin{subequations}\begin{eqnarray}}
{\end{eqnarray}\end{subequations}
\setlength{\arraycolsep}{\tempsep}}

\newenvironment{tapad}{\setlength{\tempsep}{\arraycolsep}
\setlength{\arraycolsep}{0.13889em}} % Ez azert kell, hogy ne hagyjon tul sok helyet az = korul.
{\setlength{\arraycolsep}{\tempsep}}

\newenvironment{eqnarr}{\setlength{\tempsep}{\arraycolsep}
\setlength{\arraycolsep}{0.13889em} % Ez azert kell, hogy ne hagyjon tul sok helyet az = korul.
\begin{eqnarray}}
{\end{eqnarray} \setlength{\arraycolsep}{\tempsep}}

\newenvironment{eqnarr*}{\setlength{\tempsep}{\arraycolsep}
\setlength{\arraycolsep}{0.13889em} % Ez azert kell, hogy ne hagyjon tul sok helyet az = korul.
\begin{eqnarray*}}
{\end{eqnarray*} \setlength{\arraycolsep}{\tempsep}}


%\usepackage[active]{srcltx} % SRC Specials: DVI [Inverse] Search
% Fuzz --- -------------------------------------------------------
\hfuzz5pt % Don't bother to report over-full boxes < 5pt
\vfuzz5pt % Don't bother to report over-full boxes < 5pt
% THEOREMS -------------------------------------------------------
% MATH -----------------------------------------------------------
\newcommand{\norm}[1]{\left\Vert#1\right\Vert}
\newcommand{\abs}[1]{\left\vert#1\right\vert}
\newcommand{\set}[1]{\left\{#1\right\}}
\newcommand{\Real}{\mathbb R}
\newcommand{\eps}{\varepsilon}
\newcommand{\To}{\longrightarrow}
\newcommand{\BX}{\mathbf{B}(X)}
\newcommand{\A}{\mathcal{A}}


\newcommand{\lesson}{\includegraphics[height=\baselineskip]{abrak/lightbulb.png} Mit tanultunk?}

\newcommand{\directory}{./exhibits}
\newcommand*{\newtitle}{\egroup\begin{frame}\frametitle}

\newcommand{\widefigure}[2]{\begin{frame}\frametitle{\hyperlink{#1back}{#2}}\hypertarget{#1}{{\begin{center}\includegraphics[width=\linewidth]{\directory/#1}\end{center}}}\end{frame}}
\newcommand{\longfigure}[2]{\begin{frame}\frametitle{\hyperlink{#1back}{#2}}\hypertarget{#1}{{\begin{center}\includegraphics[height=0.8\textheight]{\directory/#1}\end{center}}}\end{frame}}
%\newcommand{\fullpagefigure}[2]{\begin{frame}\frametitle{\hyperlink{#1back}{#2}}\hypertarget{#1}{{\begin{centering}$#1$\end{centering}}}\end{frame}}
\newcommand{\answer}[1]{\begin{itemize}\item #1\end{itemize}}




\newcommand{\jumpto}[2]{\hypertarget{#1back}{\hyperlink{#1}{#2}}}
\newcommand{\backto}[2]{\hypertarget{#1}{\hyperlink{#1back}{#2}}}


\title{Challenges of multidimensional transactional data}

\author{Miklós Koren\\
\#istandwithceu}
% - Give the names in the same order as the appear in the paper.
% - Use the \inst{?} command only if the authors have different
%   affiliation.


\date % (optional, should be abbreviation of conference name)
{EEA Research Committee Session}
% - Either use conference name or its abbreviation.
% - Not really informative to the audience, more for people (including
%   yourself) who are reading the slides online

%\subject{Theoretical Computer Science}
% This is only inserted into the PDF information catalog. Can be left
% out.



% If you have a file called "university-logo-filename.xxx", where xxx
% is a graphic format that can be processed by latex or pdflatex,
% resp., then you can add a logo as follows:

\pgfdeclareimage[height=0.5cm]{university-logo}{frblogo}
%\logo{\pgfuseimage{university-logo}}



% Delete this, if you do not want the table of contents to pop up at
% the beginning of each subsection:
\AtBeginSection[]
{
  \begin{frame}[plain]
    \frametitle{\color{red}\insertsection}
    \addtocounter{framenumber}{-1}
    %\tableofcontents[currentsection,currentsubsection]
  \end{frame}
}


% If you wish to uncover everything in a step-wise fashion, uncomment
% the following command:

%\beamerdefaultoverlayspecification{<+->}

\setbeamertemplate{navigation symbols}{}
\setbeamertemplate{footline}{{}\hfill\insertframenumber}
\setbeamertemplate{blocks}[rounded][shadow=true]
\setbeamercolor{block title}{bg=white!95!black}
\setbeamercolor{block body}{bg=white!95!black}

\begin{document}

\begin{frame}[plain]
  \titlepage
    \addtocounter{framenumber}{-1}
\end{frame}






\section{Representing transactional data}\hypertarget{Representing transactional data}{}
\begin{frame}\frametitle{What is transactional data?}\hypertarget{What is transactional data?}{}
\begin{itemize}
\item Many observational datasets are transactional: 
\begin{itemize}
\item administrative: customs declarations, VAT/sales tax declarations, wage data

\item private sector: sales, customer service events, website logs


\end{itemize}

\end{itemize}
\end{frame}



\begin{frame}\frametitle{Star schema}\hypertarget{Star schema}{}
\begin{block}{Dimension}\hypertarget{Dimension}{}
\begin{itemize}
\item An attribute \emph{identifying} the transaction. 

\item Typically categorical: salesperson, client, region, product.

\item But: time, space.


\end{itemize}
\end{block}
\begin{block}{Fact}\hypertarget{Fact}{}
\begin{itemize}
\item An attribute \emph{characterizing} the transaction. 

\item Typically numerical: quantity, price, freight charge.


\end{itemize}
\end{block}
\end{frame}




\longfigure{star-schema}{Star schema in a relational database}


\begin{frame}\frametitle{An econometrician's view}\hypertarget{An econometrician's view}{}
\[
X_{ijklmnop}
\]
\begin{itemize}
\item dimensions: $i,j,k,l,m,n,o,p$

\item fact: $X$




\end{itemize}
\end{frame}







\section{Real-world examples}\hypertarget{Real-world examples}{}
\begin{frame}\frametitle{Real-world examples}\hypertarget{Real-world examples}{}
\begin{itemize}
\item Product-level export (U.S.): Armenter and Koren (2013)

\item VAT (Belgium): Dhyne, Magerman and Rubinova (2015)

\item Procurement (Hungary): Koren, Szeidl, Szucs and Vedres (2017)




\end{itemize}
\end{frame}



\begin{frame}\frametitle{Product-level export (U.S.)}\hypertarget{Product-level export (U.S.)}{}
\begin{itemize}
\item Transaction: product line on a customs declaration

\item Observations: 22 million/year

\item Dimensions:
\begin{itemize}
\item Products: 9,000 Schedule-B codes

\item Exporting firms: 160,000

\item Dates: 365 days

\item Destination countries: 200
\end{itemize}

\item Combinations of dimensions: 100 trillion

\item Fraction of zeros: 99.999978\%


\end{itemize}
\end{frame}



\begin{frame}\frametitle{VAT (Belgium)}\hypertarget{VAT (Belgium)}{}
\begin{itemize}
\item Transaction: B2B sales (partner-specific VAT declaration)

\item Observations: 15 million/year

\item Dimensions:
\begin{itemize}
\item Buying firms: 2.7 million

\item Selling firms: 2.7 million
\end{itemize}

\item Combinations of dimensions: 7.3 trillion

\item Fraction of zeros: 99.999795\%


\end{itemize}
\end{frame}



\begin{frame}\frametitle{Procurement (Hungary)}\hypertarget{Procurement (Hungary)}{}
\begin{itemize}
\item Transaction: Public procurement tender

\item Observations: 20,000/year

\item Dimensions:
\begin{itemize}
\item Products: 5,900 9-digit CPV codes

\item Buying firms: 7,700

\item Selling firms: 24,000

\item Dates: 365 days
\end{itemize}

\item Combinations of dimensions: 400 trillion

\item Fraction of zeros: 99.99999999\%










\end{itemize}
\end{frame}







\section{Modeling transactional data}\hypertarget{Modeling transactional data}{}
\begin{frame}\frametitle{Two approaches to statistical modeling}\hypertarget{Two approaches to statistical modeling}{}
\begin{block}{Dimensions first}\hypertarget{Dimensions first}{}
\[
X_{ijklmnop} \sim F()
\]
independently across dimensions
\pause


\end{block}
\begin{block}{Transactions first}\hypertarget{Transactions first}{}
\[
\{X,i,j,k,l,m,n,o,p\} \sim F()
\]
independently across transactions


\end{block}
\end{frame}



\begin{frame}\frametitle{Challenges for estimation, inference and prediction}\hypertarget{Challenges for estimation, inference and prediction}{}
\begin{enumerate}\setcounter{enumi}{0}
\item Too many dimensions

\item Too many observations

\item Too many zeros

\item Too many fixed effects
\pause



\item Continuous dimensions


\end{enumerate}
\end{frame}



\begin{frame}\frametitle{Too many dimensions}\hypertarget{Too many dimensions}{}
\begin{itemize}
\item Challenging to estimate fixed effects.

\item (Within transformation can be applied if balanced.)


\end{itemize}
\end{frame}



\begin{frame}\frametitle{Too many observations}\hypertarget{Too many observations}{}
\begin{itemize}
\item Computational constraints: memory, time.

\item Common approach: arbitrary sample (e.g., zoom in on positive flows)
\begin{itemize}
\item Unknown statistical properties.


\end{itemize}

\end{itemize}
\end{frame}



\begin{frame}\frametitle{Too many zeros}\hypertarget{Too many zeros}{}
\begin{itemize}
\item In typical transactional data, more than 99.999\% of potential categories have $n=0$.

\item Multi-level modeling of zero and non-zero facts.
\begin{itemize}
\item Particularly challenging with fixed effects.
\end{itemize}

\item Endangers numerical accuracy.

\item Prediction is hard.


\end{itemize}
\end{frame}



\begin{frame}\frametitle{Too many fixed effects}\hypertarget{Too many fixed effects}{}
\begin{itemize}
\item It is common to include fixed effects for each dimension.

\item This becomes prohibitive with 4-5 dimensions and trillions of fixed effects to estimate.

\item Particularly with nonlinear estimators.


\end{itemize}
\end{frame}



\begin{frame}\frametitle{Continuous dimensions}\hypertarget{Continuous dimensions}{}
\begin{itemize}
\item Some dimensions are continuous: time, space.

\item Common approach: discretize (year, month, city, ZIP-code).
\begin{itemize}
\item Arbitrary interval definitions (see: Modifiable Area Unit Problem).

\item Independence assumption may not be valid.

\item Unnecessary duplication of data (memory, time).






\end{itemize}

\end{itemize}
\end{frame}







\section{What can we do?}\hypertarget{What can we do?}{}
\begin{frame}\frametitle{What can we do?}\hypertarget{What can we do?}{}
\begin{block}{Estimation}\hypertarget{Estimation}{}
Use multinomial or other discrete choice model for transactional data. Nonlinear, but much fewer observations.
\pause


\end{block}
\begin{block}{Inference}\hypertarget{Inference}{}
For simple null models (e.g., independent dimensions), simulating an empirical joint distribution $F$ is easy. (Armenter and Koren, 2013)
\pause


\end{block}
\begin{block}{Prediction}\hypertarget{Prediction}{}
Empirical Bayes may handle large number of zeros well ("missing butterfly problem").


\end{block}
\end{frame}



\begin{frame}\frametitle{Conclusion}\hypertarget{Conclusion}{}
\begin{itemize}
\item Transactional data is everywhere and is very useful.

\item But also very sparse: with categories far exceeding observations.

\item Model transactions rather than dimensions.
\end{itemize}
\end{frame}







\end{document}