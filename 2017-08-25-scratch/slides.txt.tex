\documentclass[aspectratio=169,compress,mathserif]{beamer}

%\usepackage[absolute]{textpos}
%\documentclass[handout,compress,mathserif]{beamer}
%\setbeameroption{show notes}

% This file is a solution template for:

% - Talk at a conference/colloquium.
% - Talk length is about 20min.
% - Style is ornate.



% Copyright 2004 by Till Tantau <tantau@users.sourceforge.net>.
%
% In principle, this file can be redistributed and/or modified under
% the terms of the GNU Public License, version 2.
%
% However, this file is supposed to be a template to be modified
% for your own needs. For this reason, if you use this file as a
% template and not specifically distribute it as part of a another
% package/program, I grant the extra permission to freely copy and
% modify this file as you see fit and even to delete this copyright
% notice.


\mode<presentation>
{
%  \usetheme{pittsburgh}
  % or ...

  \setbeamercovered{invisible}
  % or whatever (possibly just delete it)
}


\usepackage[USenglish]{babel}
\usepackage[utf8]{inputenc}




\renewcommand{\cite}[1]{({\small #1})}


\pretolerance5000 \hyphenpenalty9999
%\setlength{\TPHorizModule}{0.5cm} \setlength{\TPVertModule}{0.5cm}
%\textblockorigin{20mm}{20mm} % start everything near the top-left corner

\newcounter{ora}
\newcounter{perc}
\newcounter{kezdoora}
\newcounter{kezdoperc}
\newcounter{percek}
\setcounter{percek}{15}
\setcounter{kezdoora}{4} % for 1.35pm as the starting time

\providecommand{\leadingzero}[1]{\ifthenelse{\value{#1}<10}{0\arabic{#1}}{\arabic{#1}}}
\providecommand{\oradisplay}[1]{\ifthenelse{\value{#1}<60}{\arabic{kezdoora}:\leadingzero{#1}}{\setcounter{perc}{\value{#1}}\addtocounter{perc}{-60}\setcounter{ora}{\value{kezdoora}}\addtocounter{ora}{1}\arabic{ora}:\leadingzero{perc}}}

\providecommand{\notes}[1]{{\tiny\textbf{Note:} #1}}

%%\renewcommand{\alert}[1]{{\color{blue}#1}}
%%%%%%%%%%%%%%%%%%%%%%%%%%%%%%%%%%%%%%%%%%%%%%%%
%% Hasznos matek makrok
%%%%%%%%%%%%%%%%%%%%%%%%%%%%%%%%%%%%%%%%%%%%%%%%

\newcommand{\QED}{{}\hfill$\Box$}
\newcommand{\intl}[4]{\int_{#1}^{#2} \! {#3} \, \mathrm d{#4}}
\newcommand{\period}{\text{.}} % Ez azert kell, mert a matek . mashogy nez ki, mint a szovege.
\newcommand{\comma}{\text{,}}  % Ez azert kell, mert a matek , mashogy nez ki, mint a szovege.
\newcommand{\dist}{\,\mathop{\operatorname{\sim\,}}\limits}
\newcommand{\D}{\,\mathop{\operatorname{d}}\!}
%\newcommand{\E}{\mathop{\operatorname{E}}\nolimits}
\newcommand{\Lag}{\mathop{\operatorname{L}}}
\newcommand{\plim}{\mathop{\operatorname{plim}}\limits_{T\to\infty}\,}
\newcommand{\CES}[3]{\mathop{\operatorname{CES}}\left(\left\{#1\right\},\left\{#2\right\},#3\right)}
\newcommand{\cestwo}[5]{\left[#1^\frac1{#5}\,#2^\frac{#5-1}{#5}+#3^\frac1{#5}\,#4^\frac{#5-1}{#5}\right]^\frac{#5}{#5-1}}
\newcommand{\cesmore}[4]{\left[\sum_{#3}#1_{#3}^\frac1{#4}\,{#2}_{#3}^\frac{#4-1}{#4}\right]^\frac{#4}{#4-1}}
\newcommand{\cesPtwo}[5]{\left[#1\,#2^{1-#5}+#3\,#4^{1-#5}\right]^\frac{1}{1-#5}}
\newcommand{\cesPmore}[4]{\left[\sum_{#3}#1_{#3}\,#2_{#3}^{1-#4}\right]^\frac{1}{1-#4}}
\newcommand{\diff}[2]{\frac{\D #1}{\D #2}}
\newcommand{\pdiff}[2]{\frac{\partial #1}{\partial #2}}
\newcommand{\convex}[2]{\lambda #1 + (1-\lambda)#2}
\newcommand{\ABS}[1]{\left| #1 \right|}
\newcommand{\suchthat}{:\hskip1em}
\newcommand{\dispfrac}[2]{\frac{\displaystyle #1}{\displaystyle #2}} % Emeletes tortekhez hasznos.

\newcommand{\diag}{\mathop{\mathrm{diag\mathstrut}}}
\newcommand{\tr}{\mathop{\mathrm{tr\mathstrut}}}
\newcommand{\E}{\mathop{\mathrm{E\mathstrut}}}
\newcommand{\Var}{\mathop{\mathrm{Var\mathstrut}}\nolimits}
\newcommand{\Cov}{\mathop{\mathrm{Cov\mathstrut}}}
\newcommand{\sgn}{\mathop{\operatorname{sgn\mathstrut}}}

\newcommand{\covmat}{\mathbf\Sigma}
\newcommand{\ones}{\mathbf 1}
\newcommand{\zeros}{\mathbf 0}
\newcommand{\BAR}[1]{\overline{#1}}

\renewcommand{\time}[1]{\addtocounter{percek}{#1}}

\newlength{\tempsep}

\newenvironment{subeqs}{\setlength{\tempsep}{\arraycolsep}
\setlength{\arraycolsep}{0.13889em} % Ez azert kell, hogy ne hagyjon tul sok helyet az = korul.
\begin{subequations}\begin{eqnarray}}
{\end{eqnarray}\end{subequations}
\setlength{\arraycolsep}{\tempsep}}

\newenvironment{tapad}{\setlength{\tempsep}{\arraycolsep}
\setlength{\arraycolsep}{0.13889em}} % Ez azert kell, hogy ne hagyjon tul sok helyet az = korul.
{\setlength{\arraycolsep}{\tempsep}}

\newenvironment{eqnarr}{\setlength{\tempsep}{\arraycolsep}
\setlength{\arraycolsep}{0.13889em} % Ez azert kell, hogy ne hagyjon tul sok helyet az = korul.
\begin{eqnarray}}
{\end{eqnarray} \setlength{\arraycolsep}{\tempsep}}

\newenvironment{eqnarr*}{\setlength{\tempsep}{\arraycolsep}
\setlength{\arraycolsep}{0.13889em} % Ez azert kell, hogy ne hagyjon tul sok helyet az = korul.
\begin{eqnarray*}}
{\end{eqnarray*} \setlength{\arraycolsep}{\tempsep}}


%\usepackage[active]{srcltx} % SRC Specials: DVI [Inverse] Search
% Fuzz --- -------------------------------------------------------
\hfuzz5pt % Don't bother to report over-full boxes < 5pt
\vfuzz5pt % Don't bother to report over-full boxes < 5pt
% THEOREMS -------------------------------------------------------
% MATH -----------------------------------------------------------
\newcommand{\norm}[1]{\left\Vert#1\right\Vert}
\newcommand{\abs}[1]{\left\vert#1\right\vert}
\newcommand{\set}[1]{\left\{#1\right\}}
\newcommand{\Real}{\mathbb R}
\newcommand{\eps}{\varepsilon}
\newcommand{\To}{\longrightarrow}
\newcommand{\BX}{\mathbf{B}(X)}
\newcommand{\A}{\mathcal{A}}


\newcommand{\lesson}{\includegraphics[height=\baselineskip]{abrak/lightbulb.png} Mit tanultunk?}

\newcommand{\directory}{./exhibits}
\newcommand*{\newtitle}{\egroup\begin{frame}\frametitle}

\newcommand{\widefigure}[2]{\begin{frame}\frametitle{\hyperlink{#1back}{#2}}\hypertarget{#1}{{\begin{center}\includegraphics[width=\linewidth]{\directory/#1}\end{center}}}\end{frame}}
\newcommand{\longfigure}[2]{\begin{frame}\frametitle{\hyperlink{#1back}{#2}}\hypertarget{#1}{{\begin{center}\includegraphics[height=0.8\textheight]{\directory/#1}\end{center}}}\end{frame}}
%\newcommand{\fullpagefigure}[2]{\begin{frame}\frametitle{\hyperlink{#1back}{#2}}\hypertarget{#1}{{\begin{centering}$#1$\end{centering}}}\end{frame}}
\newcommand{\answer}[1]{\begin{itemize}\item #1\end{itemize}}




\newcommand{\jumpto}[2]{\hypertarget{#1back}{\hyperlink{#1}{#2}}}
\newcommand{\backto}[2]{\hypertarget{#1}{\hyperlink{#1back}{#2}}}


\title{Adatorientált programozás tanítása kezdőknek}

\author{Koren Miklós\\
CEU Department of Economics and Business\\
@korenmiklos}
% - Give the names in the same order as the appear in the paper.
% - Use the \inst{?} command only if the authors have different
%   affiliation.


\date % (optional, should be abbreviation of conference name)
{Scratch @ Budapest}
% - Either use conference name or its abbreviation.
% - Not really informative to the audience, more for people (including
%   yourself) who are reading the slides online

%\subject{Theoretical Computer Science}
% This is only inserted into the PDF information catalog. Can be left
% out.



% If you have a file called "university-logo-filename.xxx", where xxx
% is a graphic format that can be processed by latex or pdflatex,
% resp., then you can add a logo as follows:

\pgfdeclareimage[height=0.5cm]{university-logo}{frblogo}
%\logo{\pgfuseimage{university-logo}}



% Delete this, if you do not want the table of contents to pop up at
% the beginning of each subsection:
\AtBeginSection[]
{
  \begin{frame}[plain]
    \frametitle{\color{red}\insertsection}
    \addtocounter{framenumber}{-1}
    %\tableofcontents[currentsection,currentsubsection]
  \end{frame}
}


% If you wish to uncover everything in a step-wise fashion, uncomment
% the following command:

%\beamerdefaultoverlayspecification{<+->}

\setbeamertemplate{navigation symbols}{}
\setbeamertemplate{footline}{{}\hfill\insertframenumber}
\setbeamertemplate{blocks}[rounded][shadow=true]
\setbeamercolor{block title}{bg=white!95!black}
\setbeamercolor{block body}{bg=white!95!black}

\begin{document}

\begin{frame}[plain]
  \titlepage
    \addtocounter{framenumber}{-1}
\end{frame}






\section{Magamról}\hypertarget{Magamról}{}
\widefigure{stencil}{Első hasznos programom}
\widefigure{keyboard.jpg}{Első haszontalan programom}
\widefigure{zipdrive}{Első beruházásom az adatforradalomba}




%= Algoritmusok és adatok =
%\widefigure{AIBreakthrough}{A mesterséges intelligencia mérföldkövei \cite{Wissner-Gross, 2016}}








\section{Programozás oktatása}\hypertarget{Programozás oktatása}{}
\begin{frame}\frametitle{Két megközelítés}\hypertarget{Két megközelítés}{}
\begin{block}{A programozás egyszerű}\hypertarget{A programozás egyszerű}{}
,,Menj előre 10 lépést!"


\end{block}
\begin{block}{A programozás bonyolult}\hypertarget{A programozás bonyolult}{}
,,Szimuláld a bayesi poszterior eloszlást GPU-n!"




%% TODO: képek?


\end{block}
\end{frame}



\begin{frame}\frametitle{Hogyan lesz valakiből adattudós?}\hypertarget{Hogyan lesz valakiből adattudós?}{}
\begin{columns}
\begin{column}{0.6\textwidth}
\begin{enumerate}\setcounter{enumi}{0}
\item \textbf<1>{Tanuld meg a matekot!}

\item \textbf<2>{Tanuld meg a statisztikát!}

\item \textbf<3>{Fejleszd a geometriai intuiciód!}

\item \textbf<4>{Tanulj meg kezelni egy statisztikai programcsomagot!}

\item \textbf<5>{Ha túl bonyolult dolgot akarsz csinálni, programozd le!}
\end{enumerate}

\end{column}
\begin{column}{0.4\textwidth}
    \begin{center}
     \includegraphics<1>[width=\textwidth]{exhibits/matrix} %
     \includegraphics<2>[width=\textwidth]{exhibits/ols-formula} %
     \includegraphics<3>[width=\textwidth]{exhibits/regression-scatter} %
     \includegraphics<4>[width=\textwidth]{exhibits/regress-stata} %
     \includegraphics<5>[width=\textwidth]{exhibits/weibull} %
    \end{center}
\end{column}
\end{columns}


\end{frame}




\longfigure{efron-hastie}{A statisztika fejlődése \cite{Efron és Hastie, 2016}}






\section{Pedig a programozás egyszerű}\hypertarget{Pedig a programozás egyszerű}{}
\begin{frame}\frametitle{Pedig a programozás egyszerű}\hypertarget{Pedig a programozás egyszerű}{}
\begin{enumerate}\setcounter{enumi}{0}
\item Programozás számok nélkül.

\item Világos célok.

\item Azonnali visszacsatolás.


\end{enumerate}
\end{frame}



\begin{frame}\frametitle{Programozás számok nélkül}\hypertarget{Programozás számok nélkül}{}
\begin{itemize}
\item Az ,,adatmesterség" (data carpentry) az adatokkal való bánás mestersége.

\item Az ,,adattudomány" (data science) előszobája.


\end{itemize}
\end{frame}



\begin{frame}\frametitle{Világos célok}\hypertarget{Világos célok}{}
\begin{itemize}
\item web scraping

\item adattisztítás






\end{itemize}
\end{frame}



\begin{frame}\frametitle{Web scraping}\hypertarget{Web scraping}{}
\begin{enumerate}\setcounter{enumi}{0}
\item Felderítés

\item Letöltés

\item Adatkinyerés

\item Adatmentés


\end{enumerate}
\end{frame}




\widefigure{scraping1}{Egy egyszerű statikus honlap}
\widefigure{scraping2}{Találjuk meg benne a struktúrát!}


\begin{frame}\frametitle{Adattisztítási példa: Címfeldolgozás}\hypertarget{Adattisztítási példa: Címfeldolgozás}{}
\begin{itemize}
\item 2,5 millió címhely (1,7 millió székhely, 900 ezer telephely)

\item Egy cím anatómiája: 1075 Budapest, Károly körút 9.




\end{itemize}
\end{frame}



\begin{frame}\frametitle{Hibás irányítószám}\hypertarget{Hibás irányítószám}{}
\alert{1052} Budapest, Kossuth Lajos tér 3.
\end{frame}



\begin{frame}\frametitle{Részleges utcanév}\hypertarget{Részleges utcanév}{}
1055 Budapest, \alert{Kossuth} tér 3.
\end{frame}



\begin{frame}\frametitle{Kétértelműség}\hypertarget{Kétértelműség}{}
2700 Cegléd, Kossuth \alert{Lajos} utca 5.
\pause


2700 Cegléd, Kossuth \alert{Ferenc} utca 5.
\end{frame}



\begin{frame}\frametitle{Elírás}\hypertarget{Elírás}{}
1151 Budapest, \alert{Kosut} utca 7.
\end{frame}



\begin{frame}\frametitle{Feldolgozhatatlan mezők}\hypertarget{Feldolgozhatatlan mezők}{}
6600 Szentes, Ipari park \alert{hrsz. 3967/3.}
\pause


1093 Budapest, \alert{(Pólus Irodaház),} Lónyai utca 15.


\end{frame}







\section{Interaktív adattisztító alkalmazások}\hypertarget{Interaktív adattisztító alkalmazások}{}

\longfigure{trifacta}{Trifacta}
\longfigure{tamr}{Tamr}
\longfigure{openrefine}{OpenRefine}


\begin{frame}\frametitle{Azonnali visszacsatolás}\hypertarget{Azonnali visszacsatolás}{}
\begin{itemize}
\item interaktív IDE

\item csoportmunka, pair programming

\item játékos feladatok


\end{itemize}
\end{frame}




\longfigure{mosoly}{Együtt programozni jó}
\longfigure{codebug}{Sikerélmény}








\section{Összefoglaló}\hypertarget{Összefoglaló}{}
\begin{frame}\frametitle{Összefoglaló}\hypertarget{Összefoglaló}{}
\begin{block}{Programozni jó!}\hypertarget{Programozni jó!}{}
\begin{itemize}
\item A programozásnak legyen kézzel fogható célja!

\item Kezdjük minél korábban!

\item Programozzunk akár számok nélkül!


\end{itemize}
\end{block}
\begin{block}{Adatozás a CEU-n}\hypertarget{Adatozás a CEU-n}{}
\begin{itemize}
\item MSc in Business Analytics

\item Data Analysis for Business and Policy

\item Data @ CEU


\end{itemize}
\end{block}
\begin{block}{Kapcsolat}\hypertarget{Kapcsolat}{}
\begin{itemize}
\item economics.ceu.edu

\item twitter.com/korenmiklos


\end{itemize}
\end{block}
\end{frame}







\end{document}