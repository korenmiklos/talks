\documentclass[aspectratio=169,compress,mathserif]{beamer}


\mode<presentation>
{
%  \usetheme{pittsburgh}
  % or ...

  \setbeamercovered{invisible}
  % or whatever (possibly just delete it)
}


\usepackage[USenglish]{babel}
\usepackage[utf8]{inputenc}
\usepackage{ifthen,array}



\pretolerance5000 \hyphenpenalty9999
%\setlength{\TPHorizModule}{0.5cm} \setlength{\TPVertModule}{0.5cm}
%\textblockorigin{20mm}{20mm} % start everything near the top-left corner

\newcounter{ora}
\newcounter{perc}
\newcounter{kezdoora}
\newcounter{kezdoperc}
\newcounter{percek}
\setcounter{percek}{15}
\setcounter{kezdoora}{4} % for 1.35pm as the starting time

\providecommand{\leadingzero}[1]{\ifthenelse{\value{#1}<10}{0\arabic{#1}}{\arabic{#1}}}
\providecommand{\oradisplay}[1]{\ifthenelse{\value{#1}<60}{\arabic{kezdoora}:\leadingzero{#1}}{\setcounter{perc}{\value{#1}}\addtocounter{perc}{-60}\setcounter{ora}{\value{kezdoora}}\addtocounter{ora}{1}\arabic{ora}:\leadingzero{perc}}}

\providecommand{\notes}[1]{{\tiny\textbf{Note:} #1}}
%%%%%%%%%%%%%%%%%%%%%%%%%%%%%%%%%%%%%%%%%%%%%%%%
%% Hasznos matek makrok
%%%%%%%%%%%%%%%%%%%%%%%%%%%%%%%%%%%%%%%%%%%%%%%%

\newcommand{\QED}{{}\hfill$\Box$}
\newcommand{\intl}[4]{\int_{#1}^{#2} \! {#3} \, \mathrm d{#4}}
\newcommand{\period}{\text{.}} % Ez azert kell, mert a matek . mashogy nez ki, mint a szovege.
\newcommand{\comma}{\text{,}}  % Ez azert kell, mert a matek , mashogy nez ki, mint a szovege.
\newcommand{\dist}{\,\mathop{\operatorname{\sim\,}}\limits}
\newcommand{\D}{\,\mathop{\operatorname{d}}\!}
%\newcommand{\E}{\mathop{\operatorname{E}}\nolimits}
\newcommand{\Lag}{\mathop{\operatorname{L}}}
\newcommand{\plim}{\mathop{\operatorname{plim}}\limits_{T\to\infty}\,}
\newcommand{\CES}[3]{\mathop{\operatorname{CES}}\left(\left\{#1\right\},\left\{#2\right\},#3\right)}
\newcommand{\cestwo}[5]{\left[#1^\frac1{#5}\,#2^\frac{#5-1}{#5}+#3^\frac1{#5}\,#4^\frac{#5-1}{#5}\right]^\frac{#5}{#5-1}}
\newcommand{\cesmore}[4]{\left[\sum_{#3}#1_{#3}^\frac1{#4}\,{#2}_{#3}^\frac{#4-1}{#4}\right]^\frac{#4}{#4-1}}
\newcommand{\cesPtwo}[5]{\left[#1\,#2^{1-#5}+#3\,#4^{1-#5}\right]^\frac{1}{1-#5}}
\newcommand{\cesPmore}[4]{\left[\sum_{#3}#1_{#3}\,#2_{#3}^{1-#4}\right]^\frac{1}{1-#4}}
\newcommand{\diff}[2]{\frac{\D #1}{\D #2}}
\newcommand{\pdiff}[2]{\frac{\partial #1}{\partial #2}}
\newcommand{\convex}[2]{\lambda #1 + (1-\lambda)#2}
\newcommand{\ABS}[1]{\left| #1 \right|}
\newcommand{\suchthat}{:\hskip1em}
\newcommand{\dispfrac}[2]{\frac{\displaystyle #1}{\displaystyle #2}} % Emeletes tortekhez hasznos.

\newcommand{\diag}{\mathop{\mathrm{diag\mathstrut}}}
\newcommand{\tr}{\mathop{\mathrm{tr\mathstrut}}}
\newcommand{\E}{\mathop{\mathrm{E\mathstrut}}}
\newcommand{\Var}{\mathop{\mathrm{Var\mathstrut}}\nolimits}
\newcommand{\Cov}{\mathop{\mathrm{Cov\mathstrut}}}
\newcommand{\sgn}{\mathop{\operatorname{sgn\mathstrut}}}

\newcommand{\covmat}{\mathbf\Sigma}
\newcommand{\ones}{\mathbf 1}
\newcommand{\zeros}{\mathbf 0}
\newcommand{\BAR}[1]{\overline{#1}}

\renewcommand{\time}[1]{\addtocounter{percek}{#1}}

\newlength{\tempsep}

\newenvironment{subeqs}{\setlength{\tempsep}{\arraycolsep}
\setlength{\arraycolsep}{0.13889em} % Ez azert kell, hogy ne hagyjon tul sok helyet az = korul.
\begin{subequations}\begin{eqnarray}}
{\end{eqnarray}\end{subequations}
\setlength{\arraycolsep}{\tempsep}}

\newenvironment{tapad}{\setlength{\tempsep}{\arraycolsep}
\setlength{\arraycolsep}{0.13889em}} % Ez azert kell, hogy ne hagyjon tul sok helyet az = korul.
{\setlength{\arraycolsep}{\tempsep}}

\newenvironment{eqnarr}{\setlength{\tempsep}{\arraycolsep}
\setlength{\arraycolsep}{0.13889em} % Ez azert kell, hogy ne hagyjon tul sok helyet az = korul.
\begin{eqnarray}}
{\end{eqnarray} \setlength{\arraycolsep}{\tempsep}}

\newenvironment{eqnarr*}{\setlength{\tempsep}{\arraycolsep}
\setlength{\arraycolsep}{0.13889em} % Ez azert kell, hogy ne hagyjon tul sok helyet az = korul.
\begin{eqnarray*}}
{\end{eqnarray*} \setlength{\arraycolsep}{\tempsep}}


%\usepackage[active]{srcltx} % SRC Specials: DVI [Inverse] Search
% Fuzz --- -------------------------------------------------------
\hfuzz5pt % Don't bother to report over-full boxes < 5pt
\vfuzz5pt % Don't bother to report over-full boxes < 5pt
% THEOREMS -------------------------------------------------------
% MATH -----------------------------------------------------------
\newcommand{\norm}[1]{\left\Vert#1\right\Vert}
\newcommand{\abs}[1]{\left\vert#1\right\vert}
\newcommand{\set}[1]{\left\{#1\right\}}
\newcommand{\Real}{\mathbb R}
\newcommand{\eps}{\varepsilon}
\newcommand{\To}{\longrightarrow}
\newcommand{\BX}{\mathbf{B}(X)}
\newcommand{\A}{\mathcal{A}}




\newcommand{\directory}{./exhibits}
\newcommand*{\newtitle}{\egroup\begin{frame}\frametitle}

\newcommand{\fullpagefigure}[2]{\begin{frame}\frametitle{\hyperlink{#1back}{#2}}\hypertarget{#1}{{\begin{center}\includegraphics[height=0.9\textheight]{\directory/#1}\end{center}}}\end{frame}}
\newcommand{\widefigure}[2]{\begin{frame}\frametitle{\hyperlink{#1back}{#2}}\hypertarget{#1}{{\begin{center}\includegraphics[width=\linewidth]{\directory/#1}\end{center}}}\end{frame}}
\newcommand{\longfigure}[2]{\begin{frame}\frametitle{\hyperlink{#1back}{#2}}\hypertarget{#1}{{\begin{center}\includegraphics[height=0.8\textheight]{\directory/#1}\end{center}}}\end{frame}}
\newcommand{\widetable}[2]{\begin{frame}\frametitle{\hyperlink{#1back}{#2}}\hypertarget{#1}{{\begin{center}\includegraphics[width=\linewidth]{tables/#1}\end{center}}}\end{frame}}
\newcommand{\longtable}[2]{\begin{frame}\frametitle{\hyperlink{#1back}{#2}}\hypertarget{#1}{{\begin{center}\includegraphics[height=0.8\textheight]{tables/#1}\end{center}}}\end{frame}}
\newcommand{\answer}[1]{\begin{itemize}\item #1\end{itemize}}


\newcommand{\jumpto}[2]{\hypertarget{#1back}{\hyperlink{#1}{#2}}}
\newcommand{\backto}[2]{\hypertarget{#1}{\hyperlink{#1back}{#2}}}


\title{The distributional impact of international trade}

\author{Miklós Koren\\Central European University\\MTA KRTK and CEPR}
% - Give the names in the same order as the appear in the paper.
% - Use the \inst{?} command only if the authors have different
%   affiliation.


\date % (optional, should be abbreviation of conference name)
{JRC Science Lecture\\September 20, 2017}
% - Either use conference name or its abbreviation.
% - Not really informative to the audience, more for people (including
%   yourself) who are reading the slides online

%\subject{Theoretical Computer Science}
% This is only inserted into the PDF information catalog. Can be left
% out.



% If you have a file called "university-logo-filename.xxx", where xxx
% is a graphic format that can be processed by latex or pdflatex,
% resp., then you can add a logo as follows:

\pgfdeclareimage[height=0.5cm]{university-logo}{frblogo}
%\logo{\pgfuseimage{university-logo}}



% Delete this, if you do not want the table of contents to pop up at
% the beginning of each subsection:
\AtBeginSection[]
{
  \begin{frame}[plain]
    \frametitle{\color{red}\insertsection}
    \addtocounter{framenumber}{-1}
    %\tableofcontents[currentsection,currentsubsection]
  \end{frame}
}


% If you wish to uncover everything in a step-wise fashion, uncomment
% the following command:

%\beamerdefaultoverlayspecification{<+->}

\setbeamertemplate{navigation symbols}{}
\setbeamertemplate{footline}{twitter.com/korenmiklos{}\hfill\insertframenumber}

\begin{document}

\begin{frame}[plain]
  \titlepage
    \addtocounter{framenumber}{-1}
\end{frame}






\section{Motivation}\hypertarget{Motivation}{}


\begin{frame}\frametitle{Three trends about trade and inequality}\hypertarget{Three trends about trade and inequality}{}
\begin{enumerate}\setcounter{enumi}{0}
\item Global trade over GDP is higher now than ever.

\item Share of low-wage countries in exports has increased.

\item Within-country income inequality is on the rise.


\end{enumerate}
\end{frame}




\widefigure{openness-1830}{We are in the second wave of globalization (Federico and Tena-Junguito, 2017)}
%\longfigure{low-wage}{Trade share of low-wage countries has been increasing (Federico and Tena-Junguito, 2017)}
\longfigure{china-share}{Share of Chinese imports have increased sharply}
%\longfigure{global-inequality}{}
\longfigure{highschool-premium}{High-school and college wage premia have increased (Goldin and Katz, 2008)}
\longfigure{top10-share}{Share of top 10\% in U.S.~market income (Bourguignon, 2015)}


\begin{frame}\frametitle{Debates on trade policy}\hypertarget{Debates on trade policy}{}
\begin{block}{By 2008}\hypertarget{By 2008}{}
\pause


\end{block}
\begin{block}{After 2008}\hypertarget{After 2008}{}
\begin{itemize}
\item How does competition from low-wage countries affect US and EU workers?

\item What will be the effect of Brexit?

\item Should NAFTA be scrapped?

\item Who supports TTIP?


\end{itemize}
\end{block}
\end{frame}



\begin{frame}\frametitle{Outline}\hypertarget{Outline}{}
\begin{enumerate}\setcounter{enumi}{0}
\item A bird's eye view on trade

\item Why distribution matters

\item How can theory guide us?

\item Open questions


\end{enumerate}
\end{frame}







\section{A bird's eye view on trade}\hypertarget{A bird's eye view on trade}{}
\begin{frame}\frametitle{Each country gains from trade}\hypertarget{Each country gains from trade}{}
``If a foreign country can supply us with a commodity cheaper than we ourselves can make it, better buy it of them with some part of the produce of our own industry, employed in a way in which we have some advantage." (Smith, 1776)


\end{frame}




\widefigure{PPF1}{Technology and endowment define the production possibilities set}
\widefigure{PPF2}{Market exchange expands these possibilities}


\begin{frame}\frametitle{The technology analogy}\hypertarget{The technology analogy}{}
In the standard trade model, opening up to trade is \emph{identical} to discovering a new technology, exchanging exports for imports.
%% rhetorical device: who would oppose a new technology?
%% again, careful about redistribution




\end{frame}



\begin{frame}\frametitle{How large are these aggregate gains?}\hypertarget{How large are these aggregate gains?}{}
In absence of randomized control trials for trade policy, we can rely on natural experiments, in which trading opportunities changed suddenly, while tastes and technologies remained the same.
\begin{enumerate}\setcounter{enumi}{0}
\item Jeffersonian self embargo of U.S. trade (1808-09)

\item Meiji restoration in Japan (1859-75)

\item Closure of the Suez Canal (1967-75)

\item The age of aviation (1960-95)


\end{enumerate}
\end{frame}



\begin{frame}\frametitle{Aggregate loss from halving trade}\hypertarget{Aggregate loss from halving trade}{}
\begin{tabular}{llc}
Period & Geography      & Percentage GDP loss \\
                &                       & from halving trade    \\
\hline
1808--09 & USA & 2--3\% \\
1854--75 & Japan & 3--5\% \\
1967--75 & Europe--Asia & 10--16\% \\
1960--95 & World & 25--35\% \\
\end{tabular}


\bigskip
{\footnotesize Based on 
Irwin (2005),
Bernhofen and Brown (2004, 2005),
Feyrer (2009a, b).}




\end{frame}







\section{Why distribution matters}\hypertarget{Why distribution matters}{}
\widefigure{google-books}{Interest in inequality is increasing (Google Books)}


\begin{frame}\frametitle{Attitudes towards trade}\hypertarget{Attitudes towards trade}{}
``How much do you agree or disagree with the following statement: (Respondent’s country) should limit the import of foreign products in order to protect its national economy?"


\begin{itemize}
\item Mayda and Rodrik (2005): 55\% agree or strongly agree.

\item Meaningful correlations across countries and respondents.


\end{itemize}
\end{frame}




%\longfigure{attitude-policy}{People like trade more in more open countries (Mayda and Rodrik, 2005)}
\longfigure{attitude-education}{More educated repondents are more pro-trade in rich countries (Mayda and Rodrik, 2005)}


\begin{frame}\frametitle{New data is crucial to study distributional effects}\hypertarget{New data is crucial to study distributional effects}{}
\begin{itemize}
\item Firm-level data from balance sheets, earnings statements, customs records or
surveys have become increasingly available in a number of countries.

\item The emergence of linked employer-employee datasets (LEEDs) enables studying worker-level outcomes, such as wages
and employment probabilities.
\begin{itemize}
\item Primary source: reuse of administrative data such as social security records. 
\pause


\end{itemize}

\item But: data collection is \emph{fragmented} across countries and access for researchers is \emph{ad hoc}.




\end{itemize}
\end{frame}







\section{How can theory guide us?}\hypertarget{How can theory guide us?}{}


\begin{frame}\frametitle{Redistributive effects are not new}\hypertarget{Redistributive effects are not new}{}
``If corn can be imported cheaper than it can be grown on this rather better land, rent will again fall and profits rise, and another and better description of land will now be cultivated for profits only." (Ricardo, 1815)
\pause


\bigskip
``Australia has a small population and an abundant supply of land, much of it not very
fertile. Land is consequently cheap and wages high, in relation to most other countries.
[...] Thus trade increases
the price of land in Australia and lowers it in Europe, while tending to keep wages down
in Australia and up in Europe." (Ohlin, 1924)


\end{frame}



\begin{frame}\frametitle{Basic mechanism}\hypertarget{Basic mechanism}{}
\begin{itemize}
\item Trade changes relative prices of sectors.

\item Lower prices put competitive pressure of sectors, regions, firms, workers.

\item Their income may shrink as a result.


% == ==
% \longfigure{D_MR_12}{More efficient firms earn disproportionately higher profits (Melitz and Ottaviano, 2008)}
% \longfigure{Profits_oper_trade}{Firm profits become more unequal after trade opening (Melitz and Ottaviano, 2008)}








\end{itemize}
\end{frame}



\begin{frame}\frametitle{A recipe for analyzing distributional effects}\hypertarget{A recipe for analyzing distributional effects}{}
\begin{enumerate}\setcounter{enumi}{0}
\item Identify groups differentially affected by the policy change.

\item Measure how costly it is to switch groups.

\item Look for complementary changes. 
%% relative vs absolute effect


\end{enumerate}
\end{frame}



\begin{frame}\frametitle{Distributional effects of importing on worker income}\hypertarget{Distributional effects of importing on worker income}{}
\begin{tabular}{llll}
Grouping & Exposure & Evidence  \\
\hline
Sector & Sectoral tariff rates & Canada, USA, EU (large),\\
 && Colombia (small) \\
Region & Import competing sectors & India, USA (large) \\
Worker/firm & Offshoring & Denmark (small), Indonesia (large) \\
\end{tabular}


\bigskip
{\footnotesize Based on 
Trefler (2004),
Pierce and Schott (2015),
Attanasio, Goldberg and Pavcnik (2004),
Goldberg and Pavcnik (2005),
Topalova (2010),
Autor, Dorn, Hanson and Song (2014),
Hakobyan and McLaren (2016),
Hummels, Jorgensen, Munch ans Xiang (2014),
Kasahara, Liand and Rodrigue (2015).}


\end{frame}



\begin{frame}\frametitle{Distributional effects of \emph{exporting} on income}\hypertarget{Distributional effects of \emph{exporting} on income}{}
\begin{tabular}{llll}
Grouping & Exposure & Evidence  \\
\hline
Region & Liberalized export sectors & Vietnam (large)\\
Firm & Productive \emph{v} unproductive & Chile + many others (small) \\
Worker/firm & Exporting \emph{v} non-exporting & Argentina, Mexico, Denmark (small)  \\
\end{tabular}


\bigskip
{\footnotesize Based on 
McCaig (2011),
Pavcnik (2002),
Verhoogen (2008),
Brambilla, Lederman and Porto (2012).}




\end{frame}



\begin{frame}\frametitle{Costs of switching are large}\hypertarget{Costs of switching are large}{}
\begin{itemize}
\item Very large costs of switching industry in the U.S.: equivalent to 4--13 years of wage income (Artuç, Chaudhuri, and McLaren, 2010).

\item Smaller, but diverse switching costs in Brazil, but worker reallocation can take 9--30 years (Dix-Carneiro, 2014).

\item Regional labor-market effects of trade liberalization peak after 20 years in Brazil (Dix-Carniero and Kovak, 2015).


\end{itemize}
\end{frame}



\begin{frame}\frametitle{Complementary effects}\hypertarget{Complementary effects}{}
\begin{itemize}
\item As firms offshore, they become more competitive and can expand in scale (Grossman and Rossi-Hansberg, 2008).

\item This can (partly) offset the negative effect on employment and wages.


% == ==
% \longfigure{imported-inputs-5}{Demand for domestic intermediate inputs is not too sensitive to tariffs (Halpern, Koren and Szeidl, 2015)}


\end{itemize}
\end{frame}



\begin{frame}\frametitle{Chinese imports are beneficial for US \emph{firms}}\hypertarget{Chinese imports are beneficial for US \emph{firms}}{}
\begin{itemize}
\item Even if some establishments shrink in response to Chinese imports, firms \emph{as a whole} increase (Magyari, 2017)
\begin{itemize}
\item manufacturing employment

\item complementary services

\item production wages


\end{itemize}

\end{itemize}
\end{frame}







\section{Open questions}\hypertarget{Open questions}{}
\begin{frame}\frametitle{Open questions}\hypertarget{Open questions}{}
\begin{enumerate}\setcounter{enumi}{0}
\item Complementarity between trade and technology

\item New research designs based on new data


\end{enumerate}
\end{frame}



\begin{frame}\frametitle{Complementarity between trade and technology}\hypertarget{Complementarity between trade and technology}{}
\begin{itemize}
\item Most studies focus on the competitive effect of trade: prices adjust, firms expand/shrink, worker income adjusts.

\item In these explanations, trade liberalization and technical progress are alternative competitive forces. 

\item New approaches suggest complementarities between the two:
\begin{itemize}
\item trading as an activity, quality needed to export (Hallak and Sivadasan, 2013, Boler, Moxnes and Ulltveit-Moe, 2015)

\item using imported technology (Koren and Csillag, 2017, Halpern, Hornok, Koren and Szeidl, 2017)


\end{itemize}

\end{itemize}
\end{frame}



\begin{frame}\frametitle{New research designs based on new data}\hypertarget{New research designs based on new data}{}
\begin{itemize}
\item To study the heterogeneous effects of policy, micro data is needed on firms and workers.

\item These are often collected outside traditional statistical agencies:
\begin{itemize}
\item administrative data (social security, VAT filings)

\item business data (financials, transactions data, location tracking)
\end{itemize}

\item Useful to analyze
\begin{itemize}
\item full impact of policy (earnings, job loss, transfers, job transitions)

\item international linkages

\item long-run effects


\end{itemize}

\end{itemize}
\end{frame}



\begin{frame}\frametitle{Needed}\hypertarget{Needed}{}
\begin{block}{A Manifesto for Economic Research in Europe (COEURE, 2016)}\hypertarget{A Manifesto for Economic Research in Europe (COEURE, 2016)}{}
\begin{enumerate}\setcounter{enumi}{0}
\item Facilitate data access for researchers

\item Improve data design and data harmonisation

\item Support economic data infrastructure in Europe


\end{enumerate}
\end{block}
\end{frame}







\section{Conclusion}\hypertarget{Conclusion}{}
\begin{frame}\frametitle{Conclusion}\hypertarget{Conclusion}{}
\begin{enumerate}\setcounter{enumi}{0}
\item Trade \emph{always} redistributes income.

\item Recent advances in measurement made it possible to identify losers.

\item Improved data access is needed to quantify the losses.
\end{enumerate}
\end{frame}







\end{document}