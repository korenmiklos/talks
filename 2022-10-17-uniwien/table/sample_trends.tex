\begin{tabular}{rrrrr}
\hline \hline
 Year &  Workers &  Firms & \begin{tabular}[x]{@{}c@{}}Fraction importing\\(percent)\end{tabular} &  \begin{tabular}[x]{@{}c@{}}Import exposure\\(percent)\end{tabular}\\
\hline

1992 & 10,853 & 1,823 & 35.27 & 16.57\\
1993 & 14,185 & 2,541 & 40.10 & 22.73\\
1994 & 14,695 & 2,773 & 39.07 & 27.26\\
1995 & 15,750 & 2,902 & 44.16 & 30.88\\
1996 & 15,419 & 2,775 & 48.74 & 34.35\\
1997 & 13,668 & 2,676 & 52.91 & 37.25\\
1998 & 15,239 & 2,754 & 55.22 & 40.04\\
1999 & 14,418 & 2,834 & 56.84 & 41.65\\
2000 & 14,805 & 2,966 & 55.89 & 43.44\\
2001 & 14,528 & 2,874 & 57.59 & 45.14\\
2002 & 15,907 & 2,345 & 53.40 & 45.99\\
2003 & 15,185 & 2,223 & 52.33 & 46.66\\
2004 & 15,261 & 2,281 & 49.86 & 46.66\\
\hline \hline
\end{tabular}
\begin{tablenotes}
\item \footnotesize  Notes: ``Fraction importing'' denotes the fraction of workers in the sample in importer occupations and importer firms ($\chi_{jot}=1$). ``Import exposure'' is defined on a balanced sample of firm-occupations and denotes the same importer fraction in this balanced sample.
\end{tablenotes}