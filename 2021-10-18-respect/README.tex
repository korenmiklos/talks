% Options for packages loaded elsewhere
\PassOptionsToPackage{unicode}{hyperref}
\PassOptionsToPackage{hyphens}{url}
%
\documentclass[
  ignorenonframetext,
  aspectratio=16,
]{beamer}
\usepackage{pgfpages}
\setbeamertemplate{caption}[numbered]
\setbeamertemplate{caption label separator}{: }
\setbeamercolor{caption name}{fg=normal text.fg}
\beamertemplatenavigationsymbolsempty
% Prevent slide breaks in the middle of a paragraph
\widowpenalties 1 10000
\raggedbottom
\setbeamertemplate{part page}{
  \centering
  \begin{beamercolorbox}[sep=16pt,center]{part title}
    \usebeamerfont{part title}\insertpart\par
  \end{beamercolorbox}
}
\setbeamertemplate{section page}{
  \centering
  \begin{beamercolorbox}[sep=12pt,center]{part title}
    \usebeamerfont{section title}\insertsection\par
  \end{beamercolorbox}
}
\setbeamertemplate{subsection page}{
  \centering
  \begin{beamercolorbox}[sep=8pt,center]{part title}
    \usebeamerfont{subsection title}\insertsubsection\par
  \end{beamercolorbox}
}
\AtBeginPart{
  \frame{\partpage}
}
\AtBeginSection{
  \ifbibliography
  \else
    \frame{\sectionpage}
  \fi
}
\AtBeginSubsection{
  \frame{\subsectionpage}
}
\usepackage{lmodern}
\usepackage{amssymb,amsmath}
\usepackage{ifxetex,ifluatex}
\ifnum 0\ifxetex 1\fi\ifluatex 1\fi=0 % if pdftex
  \usepackage[T1]{fontenc}
  \usepackage[utf8]{inputenc}
  \usepackage{textcomp} % provide euro and other symbols
\else % if luatex or xetex
  \usepackage{unicode-math}
  \defaultfontfeatures{Scale=MatchLowercase}
  \defaultfontfeatures[\rmfamily]{Ligatures=TeX,Scale=1}
\fi
% Use upquote if available, for straight quotes in verbatim environments
\IfFileExists{upquote.sty}{\usepackage{upquote}}{}
\IfFileExists{microtype.sty}{% use microtype if available
  \usepackage[]{microtype}
  \UseMicrotypeSet[protrusion]{basicmath} % disable protrusion for tt fonts
}{}
\makeatletter
\@ifundefined{KOMAClassName}{% if non-KOMA class
  \IfFileExists{parskip.sty}{%
    \usepackage{parskip}
  }{% else
    \setlength{\parindent}{0pt}
    \setlength{\parskip}{6pt plus 2pt minus 1pt}}
}{% if KOMA class
  \KOMAoptions{parskip=half}}
\makeatother
\usepackage{xcolor}
\IfFileExists{xurl.sty}{\usepackage{xurl}}{} % add URL line breaks if available
\IfFileExists{bookmark.sty}{\usepackage{bookmark}}{\usepackage{hyperref}}
\hypersetup{
  pdftitle={How Similar Are International Economic Relations of EU Member States? Comparing Trade, Investment and Political Behavior},
  pdfauthor={Matteo Fiorini; Miklós Koren; Filippo Santi; Gergő Závecz},
  hidelinks,
  pdfcreator={LaTeX via pandoc}}
\urlstyle{same} % disable monospaced font for URLs
\newif\ifbibliography
\setlength{\emergencystretch}{3em} % prevent overfull lines
\providecommand{\tightlist}{%
  \setlength{\itemsep}{0pt}\setlength{\parskip}{0pt}}
\setcounter{secnumdepth}{-\maxdimen} % remove section numbering
\usepackage{pgfpages}
\usepackage{microtype}
\usepackage{tikz}
  \usetikzlibrary{positioning}
  \usetikzlibrary{arrows}
  \usetikzlibrary{graphs}

\definecolor{CTred}{RGB}{229,32,32}
\definecolor{CTgrey}{RGB}{153,153,153}


% colors: white text on 90% black background
\setbeamercolor{normal text}{fg=black,bg=white}

% light blue as a highlight color
\setbeamercolor*{structure}{fg=CTred}
\setbeamercolor{section title}{fg=CTred}
\setbeamercolor{alerted text}{use=structure,fg=CTred}
\setbeamercolor*{palette primary}{use=structure,fg=structure.fg}
\setbeamercolor*{palette secondary}{use=structure,fg=structure.fg!95!black}
\setbeamercolor*{palette tertiary}{use=structure,fg=structure.fg!90!black}
\setbeamercolor*{palette quaternary}{use=structure,fg=structure.fg!95!black,bg=black!80}

\setbeamercolor*{framesubtitle}{fg=white}


% use system fonts: here, Gill Sans
\usefonttheme{professionalfonts}
\setbeamerfont{quote}{shape=\upshape}

% eliminate silly beamer navigation line at bottom of slides
\setbeamertemplate{navigation symbols}{}

% ensure text jusfication
\usepackage{ragged2e}
\justifying

% pandoc makes 2nd-lever headers into blocks, and this ensures justification
% in blocks too
\addtobeamertemplate{block begin}{}{\justifying}




\urlstyle{same}
\usepackage[overlay,absolute]{textpos}

\setbeamertemplate{items}[square]

\TPGrid[10 mm,8 mm]{9}{8}
% beamer's left and right margin is 10 mm. The top/bottom margin is ??
% or without a header ??
% the slide dimensions are 128 mm x 96 mm
% so the resulting \TPHorizModule = 12 mm and \TPVertModule = 10 mm

% uncomment if you want biblatex for citations on slides

% \usepackage{csquotes}
% \usepackage[notes,short,noibid,backend=biber]{biblatex-chicago}
% \bibliography{course.bib} 

\providecommand{\exhibit}[2]{\includegraphics[keepaspectratio, height=0.9\textheight, width=\textwidth]{assets/img/#1}\\ {\tiny #2}}

\title{How Similar Are International Economic Relations of EU Member
States? Comparing Trade, Investment and Political Behavior}
\author{Matteo Fiorini \and Miklós Koren \and Filippo Santi \and Gergő
Závecz}
\date{RESPECT Closing Conference}

\begin{document}
\frame{\titlepage}

\hypertarget{motivation}{%
\section{Motivation}\label{motivation}}

\hypertarget{data}{%
\section{Data}\label{data}}

\begin{frame}{Trade, Investments, and additional controls}
\protect\hypertarget{trade-investments-and-additional-controls}{}
Export data come from COMEXT (Eurostat 2019). We use the chapter-level
product distribution of bilateral exports between EUMS and their
Extra-EU partners, measured between 2001 and 2017 (although most
analysis uses the years 2015-17 due to unavailability of other political
measures discussed below). Because we do not have access to
shipment-level data, we approximate the number of shipments by dividing
the value of exports by EUR 12,000, following estimates in Hornok and
Koren (2015a,b).

Investment data come from the fDIMarket database (Financial Times,
2019). We aggregate single Greenfield FDI transactions at
origin-destination-sector level over the period 2003-2018, focusing on
the flows that occurred between EUMS and the rest of the world. Also in
this case, we divide the value of larger-than-average investments by the
average investment value (computed on strictly positive transactions
only). In other words, we are assuming that a factor-\(n\) larger
investment is comparable to \(n\) average size FDI. This assumption is
required to reduce data sparsity.

Although the main focus is to explore how trade and investment structure
against extra-EU partner countries affect EUMS policy efforts toward
idiosyncratic economic strategy, some specifications also include
intra-EU trade/investments flows.

We use the GeoDist dataset (Mayer and Zignago 2011) to include
geographic distance as well as historical and cultural ties. We also
include current GDP (expressed in US dollars and taken in log form),
which we take from the World Bank - World Development Indicators and the
National Accounts database of the OECD (World Bank 2020).
\end{frame}

\begin{frame}[fragile]{Political Variables}
\protect\hypertarget{political-variables}{}
To study the behavior of Member States, we turn to media mentions of
state visits and similar events. For what concern our main variable of
interest, we extract information on the number of events about economic
or diplomatic cooperation between two state actors for the period
2015-17 from the Global Database of Events, Language and Tone (The GDELT
Project 2020).

In particular, we limit our attention to positive, cooperative events
(as coded by GDELT) in which government agencies or decision makers from
EUMS are considered as the
\texttt{initiating\ actors\textquotesingle{}\textquotesingle{}.\ There\ are\ two\ groups\ of\ cooperative\ events:\ one\ classified\ as}intent'\,'
(intent to cooperate) and one classified as ``visits.'\,' In this latter
category we include state visits, formal negotiations, signing of
agreements, and material cooperation. With reference to any member
state, we tally all such events happening in, or being related to, any
potential partner country.

As an example, let us focus on a single member state, say France. Then,
the French foreign minister visiting Turkey could be one event, which
would add up to the France-Turkey bilateral record. The French president
arguing for further cooperation with Russia would instead accrue to the
France-Russia bilateral record.

Given this approach, we construct two measures of government
collaboration between any EUMS \(i\) and a given partner \(j\) in year
\(t\), denoted by \(\text{INTENT}_{ijt}\) and \(\text{VISIT}_{ijt}\).
The procedure followed to extract and manipulate information from GDELT
is described in Koren et al (2020).

In some of the models presented on Figures\textasciitilde4
to\textasciitilde7 we also control for two additional time-varying and
symmetric political variables. The first, \emph{Agreement} is the log of
the vote similarity index of two countries in a given year, and comes
from the United Nations General Assembly Voting Data (Voeten, Strezhnev
and Bailey 2009).

The \emph{Difference in democracy} comes from the Quality of Government
Basic Dataset (Teorell et al 2020), and captures the (log) absolute
difference in the imputed Freedom House Level of Democracy scores
between any two countries.
\end{frame}

\hypertarget{methods}{%
\section{Methods}\label{methods}}

\begin{frame}{Export composition}
\protect\hypertarget{export-composition}{}
The exports of country \(i\) to country \(j\) in year \(t\) is hence
characterized by a sequence of shares,
\(s_{ijt1}, s_{ijt2}, ..., s_{ijtP}\), where \(P\) is the overall number
of products, with the shares summing to one,
\(\sum_{p=1}^P s_{ijtp}=1\). These value shares completely characterize
the trade structure of a pair of countries for our purposes. We also
control for the overall volume of trade.

Comparing country \(i\) to the EU average (denoted by \(*\)) amounts to
comparing two sets of shares, \(\{s_{ijtp}\}\) and \(\{s_{*jtp}\}\). Our
goal is to ask if country \(i\)'s trade shares are different from the EU
average, and if so, to quantify the magnitude of the difference.

Industry similarity indexes between regions and countries have been
proposed in other contexts by Finger and Kreinin (1979), Krugman (1991),
and Fontagné et al (2018). In contrast to these, our proposed index of
similarity is based on an economic choice model (Anderson et al.~1992).
\end{frame}

\begin{frame}{Kullback-Leibler Divergence}
\protect\hypertarget{kullback-leibler-divergence}{}
Our preferred measure of difference between country-specific and EU
trade shares is the Kullback-Leibler divergence (Kullback 1987, KLD
henceforth), defined as \begin{equation}
    \text{KLD}_{ijt} =
    \sum_{p=1}^P
        s_{ijtp}
        \ln(s_{ijtp} / s_{*jtp}).
\end{equation} This is a measure of distance between the two
distributions, only taking the value zero if all the products have the
same share, and positive otherwise. As mentioned above, a key benefit of
this index is that it is based on utility maximizing decision model.
More specifically, take a consumer with logit preferences (a standard
assumption in discrete choice models) whose ideal consumption shares are
given by \(s_{ijt}\). If this consumer instead consumes the products in
shares \(s_{*jt}\), her utility will be reduced by a magnitude
proportional to the KLD between \(s\) and \(s*\).
\end{frame}

\begin{frame}{The problems with sparsity}
\protect\hypertarget{the-problems-with-sparsity}{}
In practice, the KLD index will never be zero, as no two countries have
exactly the same product composition of exports. In order to
quantitatively judge what constitutes a significant gap between the
trade composition of two countries, we test whether the KLD is
significantly different from zero. This is important because the KLD
index will be biased upwards in small samples.

To test for statistical significance and to mitigate small sample bias,
we conduct the following procedure. Let \(x_{ijtp}\) denote the number
of export shipments from country \(i\) to country \(j\) in product \(p\)
in year \(t\). Shipments are the basic units of observation in trade
statistics, and a small number of shipments can lead to small-sample
bias (Armenter and Koren 2014). The total number of shipments between a
pair of countries \(n_{ijt} = \sum_p x_{ijtp}\) is taken as given.
\end{frame}

\begin{frame}{The Polya Index}
\protect\hypertarget{the-polya-index}{}
The \emph{null hypothesis} is that all countries' shipments are
distributed according to the same distribution. We chose the
multivariate Polya distribution (Eggenberger and Pólya 1923) as the
parametric distribution that best suites this application.

More specifically, \(\{x_{ijtp}\}\) is assumed to be distributed
according to the Polya distribution with expected product shares
\(\{\pi_{jtp}\}\) and a precision \(T_{jt}\). We estimate the expected
shares and the precision parameter with maximul likelihood, separately
for each partner country \(j\) and year \(t\).

Under this null hypothesis, the KLD index has a distribution
\(F_{ijt}\): \begin{equation}
    \Pr(\text{KLD}_{ijt} \le x | n_{ijt}) = F_{ijt}(x).
\end{equation} Computing this distribution in closed form is possible,
but requires prohibitively many combinatorial steps. We would have to
compute the probability of each possible allocation of shipments, for
thousands of shipments. With a 100 product categories, even just 1,000
shipments could be distributed about \(10^{39}\) different ways.
\end{frame}

\begin{frame}{Computation}
\protect\hypertarget{computation}{}
Instead, we approximate \(F()\) with its empirical distribution
function. We simulate the distribution with 10,000 Monte Carlo draws and
define \(\hat F_{ijt}(x)\) as the fraction of draws in which the
simulated KLD index is smaller or equal to \(x\).

We then define the \emph{Polya Index} as the tail probability of the
empirical distribution, evaluated at the actual KLD, \begin{equation}
    \text{Polya}_{ijt} \equiv 1 - \hat F_{ijt}(\text{KLD}_{ijt}).
\end{equation}

The Polya Index captures the probability that we would observe the
measured KLD index or higher, conditional on all countries' trade
structures being the same in expectation. Formally, it is the
statistical size of a one-sided test of the null hypothesis that all
countries have a KLD of zero (with their shares generated from the Polya
distribution).
\end{frame}

\begin{frame}{Interpretation}
\protect\hypertarget{interpretation}{}
The Polya Index is an index of similarity. When the product distribution
of the country is statistically indistinguishable from the rest of the
countries, \(\text{Polya}_{ijt}\) is very close to one. By contrast, low
levels of the Polya Index mean that we can reject the null hypothesis of
similarity. It is important to note, however, that a large Polya Index
does not necessarily mean a full alignment of the country's trade
structure with that of the EU. It can also arise when we have too few
transactions to statistically differentiate the two trade structures. A
low Polya Index, on the other hand, surely indicates significant
differences.

With the due differences, the Polya Index can also be applied to
bilateral investments, although the sparsity of investment transactions
might make harder for statistically significant differences to emerge in
investment portfolios (compared to trade). Hence, we expect the Polya
Index for investment to be larger than its trade counterpart.
\end{frame}

\hypertarget{findings}{%
\section{Findings}\label{findings}}

\end{document}
